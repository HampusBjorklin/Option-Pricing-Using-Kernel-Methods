%----------------PACKAGES---------------------------------------------
\documentclass[a4paper]{article}      %Dokumentklass
\usepackage{babelbib}
\usepackage{indentfirst}
\usepackage[english, swedish]{babel}
\usepackage[T1]{fontenc}                    %Anger standardtypsnitt
\usepackage[utf8]{inputenc}                 %Definierar kodningsspråk
\usepackage[font={small,it}]{caption}		  %Storlek, italic caption
\usepackage[font={footnotesize,it}]{subcaption}%Storlek, italic caption
\usepackage{graphicx}                          %Krävs för figurer
\usepackage{mathtools}
\usepackage{algorithm}
\usepackage{algpseudocode}
\usepackage{mathrsfs}
\usepackage{mathabx}
\usepackage{tocloft}
\usepackage{booktabs, multirow}
\renewcommand{\arraystretch}{1.2}
\usepackage{epstopdf}

\usepackage{physics}
\usepackage{pgfplots} 
\pgfplotsset{width=13cm,compat=newest} 
\pgfplotsset{plot coordinates/math parser=false} 
\newlength\figureheight 
\newlength\figurewidth 
\usepackage{fullpage}                       %Mindre marginaler
\usepackage{setspace}                       %Kan öka radavstånd
\usepackage{float}                          %För att flytta figurer bl.a
\usepackage{parskip}                        %Blankrad inte indrag
\usepackage{amsmath}                        %Matematik
\usepackage{units}                          %Matematik
\usepackage{textcomp}                       %Tecken
\usepackage{titlesec}						%Formaterar section, subsec...
\usepackage[normalem]{ulem}					%Tillåter understryket
\usepackage{url}                            %För hyperlänkar
\usepackage{float}

\usepackage[toc,page]{appendix}	%Bilagor
\usepackage[final]{pdfpages} %pdf import

\usepackage[justification=centering]{caption}
\usepackage[font=small]{caption}
\usepackage{caption}

\begin{document}

Solving the Black Scholes PDE with Holger's reproducing kernel. In this example, MQ kernel is used. The two-dimensional PDE is approximated with collecting of order-one-functions. 

For $d$ dimensions, $\vec{x} \in [0\; 1]^d$.

Define the desired evaluation points $X_E = \bigl\{ \vec{x_1}, \vec{x_2}, \vec{x_3}, ..., \vec{x_{N_e}} \bigr\}$
and the set of anchored sample points  $X_N = \bigl\{ (x_1, a_y), (x_2, a_y),...,(x_n, a_y), (a_x, a_y), (a_x, y_1),...,(a_x, y_n) \bigr\}$
with $ 0 < x_i, y_i, < 1 $.
PDE, equation \ref{ekv:PDE}.

\begin{equation}
    \begin{aligned}
        &\frac{\partial u}{\partial t} + \frac{1}{2} \sum_{i, j}^d \mathbf{C}_{i,j} s_i s_j \frac{\partial^2 u}{\partial s_i \partial s_j} + \sum_{i}^d r s_i \frac{\partial u}{\partial s_i} - ru = 0 \quad &\vec{s} \in [0,1]^d, t \in [0, T] \\
        &u(\vec{s},t) = \phi(\vec{s}, t) = \mathrm{max}(\frac{1}{d}\sum^d s_i - Ke^{T-t}, 0)    &||\vec{s}|| \geq 1, t \in [0, T] \\
        &u(\vec{s}, t) = 0 &||\vec{s}|| = 0, t \in [0, T] \\
        &u(\vec{s},t) = \phi(\vec{s},T) &t = T \\
    \end{aligned}
    \label{ekv:PDE}
\end{equation}

%This is a comment
We have different sets of values.
\begin{itemize}
\item Evaluation points  $X_{eval} = \left\{ (x_i, y_i) \right\}_i ^ {N_e}$
\item Anchored center points $X = \left\{ (x_i, y_i) \right\}_i ^ {N}$ 
\item Inner points $X_{inner}\subseteq X,\left\{ (x_i, y_i) \right\}_i ^ {N_inner}$
\item Far and close points, where boundary condition apply
$X_{far}, X_{close} \subseteq X,\left\{ (x_i, y_i) \right\}_i ^ {N_{far}}, \left\{ (x_i, y_i) \right\}_i ^ {N_{close}}$
\end{itemize}
We get that $N = N_{inner} + N_{far} + N_{close}$

\section*{Reproducing kernel}
The reproducing kernel is for the 2D case defines as follows.
\begin{equation}
	\begin{gathered}
	R(\vec{x}^1, \vec{x}^2) = 1 + k({x}^1_1, {x}^2_1) + k({x}^1_2, {x}^2_2) \\
	k(x_1, x_2) = \sqrt{1 + \varepsilon^2 ||x_1 - x_2||_2^2}
	\end{gathered}
	\label{ekv:RepKernel}
\end{equation}
From equation \ref{ekv:RepKernel} we can define the following matrix.
\begin{equation}
\mathbf{A_0} = \left[R(\vec{x}^j, \vec{x}^k)\right] _{\vec{x}_j,\vec{x}_k \in X} 
\label{ekv:A0}
\end{equation}
Followingly, the derived reproducing kernels with associated matrix may be built.
\begin{equation}
	\begin{gathered}
	\partial_{x_i} R(\vec{x}^j, \vec{x}^k) = \frac{\varepsilon^2(x_i^j - x_i^k)}{\sqrt{1 + \varepsilon^2||x_i^j - x_i^k||^2}} \Rightarrow  \mathbf{A_1}, \mathbf{A_2} \\[5pt]
	\partial^2_{x_ix_i} R(\vec{x}^j, \vec{x}^k) = \frac{\varepsilon^2}{(1 + \varepsilon^2||x_i^j - x_i^k||^2)^{3/2}} \Rightarrow  \mathbf{A_{11}}, \mathbf{A_{22}} \\[5pt]
	\partial^2_{x_ix_j} R(\vec{x}^j, \vec{x}^k) = 0  \Rightarrow \mathbf{A_{12}} \label{ekv:A12}
	\end{gathered}
\end{equation}

The projected approximation of a function $f(\vec{x})$ at a arbitrary set of evaluation points (may be the same as center points).
\begin{equation}
	\underbrace{\hat{f}(X_{eval})}_{R^{Ne \times 1}} =
	\underbrace{ 
	\left[R(\vec{x}_i, \vec{x}_j)  				\right]_{\vec{x}_i \in X_{eval}, \vec{x}_j \in X} 	}_{R^{Ne \times N}}	
	\cdot 
	\underbrace{
	\mathbf{A_0}^{-1}
	}_{R^{N\times N}}
	\underbrace{ 
	f(X)
	}_{R^{N \times 1 }}
	\label{ekv:EvalForm}
	\end{equation} 

From equation \ref{ekv:EvalForm}, the way to approximate the 2D Black Scholes operator becomes.
\begin{gather}
\begin{aligned}
 \mathcal{O}_{BS} =& ( rX_1^1 \mathbf{B_1} + rX_2^2 \mathbf{B_2} \\
&+ \frac{1}{2} \sigma_1^2 (X_1^1)^2 \mathbf{B_{11}} 
+ \frac{1}{2} \sigma_2^2 (X_2^2)^2 \mathbf{B_{22}} \\
&+ \rho \sigma_1 \sigma_2 X_1^1 X_2^2 \mathbf{B_{12}} \\
& - r\mathbf{B_0} ) \mathbf{A_0}^{-1} \\
\end{aligned}
\\
\begin{cases}
\mathbf{B_0} = [\mathbf{A_0}]_{i\in N_{i},j \in N} \\
\mathbf{B_1} = [\mathbf{A_1}]_{i\in N_{i},j \in N} \\
\mathbf{B_2} = [\mathbf{A_2}]_{i\in N_{i},j \in N} \\
\mathbf{B_{11}} = [\mathbf{A_{11}}]_{i\in N_{i},j \in N} \\
\mathbf{B_{22}} = [\mathbf{A_{22}}]_{i\in N_{i},j \in N} \\
\mathbf{B_{12}} = [\mathbf{A_{12}}]_{i\in N_{i},j \in N} \\
\end{cases}
\end{gather}
\section*{Coordinate Transformation}
We want to rotate the coordinate system and solve the problem for $v_1, v_2$ instead of  $x_1, x_2$. The transformation is linear and equation \ref{ekv:2DervCoordinate} \& \ref{ekv:1DervCoordinate} is only applicable in the linear case. 
\begin{gather}
\vec{v} = \frac{1}{2}
\begin{bmatrix}
1 & 1 \\ 1 & -1
\end{bmatrix}
\vec{s} \\
\frac{\partial^2 u}{\partial {s_i}\partial {s_j}} =
\sum_{k,l = 1}^d 
\left( 
\frac{\partial^2 u}{\partial {v_k}\partial {v_l}}
\frac{\partial v_l}{\partial {s_j}}
\frac{\partial v_k}{\partial {s_i}}
\right) \label{ekv:2DervCoordinate}\\
\frac{\partial u}{\partial s_i}
= \sum_k^d
\left(
\frac{\partial u}{\partial v_k}
\frac{\partial v_k}{\partial s_i}
\right) \label{ekv:1DervCoordinate}
\end{gather}

With this, we can rewrite equation \ref{ekv:PDE} in this new system.

\begin{gather}
\begin{aligned}
 \mathcal{O}_{BS}^\mathcal{T} =& \Big( r(V_1^1 + V_2^2 - \frac{1}{2}) \mathbf{B_1} + r(V_1^1 - V_2^2 + \frac{1}{2}) \mathbf{B_2} \\
&+ \frac{1}{2} \sigma_1^2 (V_1^1 + V_2^2 - \frac{1}{2})^2 \mathbf{B_{11}} 
+ \frac{1}{2} \sigma_2^2 (V_1^1 - V_2^2 + \frac{1}{2})^2 \mathbf{B_{22}} \\
&+ \rho \sigma_1 \sigma_2 (V_1^1 + V_2^2 - \frac{1}{2}) (V_1^1 - V_2^2 + \frac{1}{2}) \mathbf{B_{12}} \\
& - r\mathbf{B_0} \Big) \mathbf{A_0}^{-1} \\
\end{aligned} \label{ekv:OTransform}
\\
\begin{cases}
X_1^1 = (V_1^1 + V_2^2 - \frac{1}{2}) \\
X_2^2 = (V_1^1 - V_2^2 + \frac{1}{2}) \\
\mathbf{B_0} = [\mathbf{A_0}]_{i\in N_{i},j \in N} \\
\mathbf{B_1} = [\frac{1}{2}(\mathbf{A_1} + \mathbf{A_2})]_{i\in N_{i},j \in N} \\
\mathbf{B_1} = [\frac{1}{2}(\mathbf{A_1} - \mathbf{A_2})]_{i\in N_{i},j \in N} \\
\mathbf{B_{11}} = [\frac{1}{4}(\mathbf{A_{11}} + 2\mathbf{A_{12}} + \mathbf{A_{22}})]_{i\in N_{i},j \in N} \\
\mathbf{B_{22}} = [\frac{1}{4}(\mathbf{A_{11}} - 2\mathbf{A_{12}} + \mathbf{A_{22}})]_{i\in N_{i},j \in N} \\
\mathbf{B_{12}} = [\frac{1}{2}(\mathbf{A_{11}} - \mathbf{A_{22}})]_{i\in N_{i},j \in N} \\
\end{cases}
\label{ekv:BTransform}
\end{gather}
\section*{Time Solver (BDF2)}
From earlier equations, we can define the system that should be solved at each timestep. Note that the set each point belongs to (Inner, Far) is still defined in terms of $x_1, x_2$. Reversing time $ \tau = T - t$ gives:
\begin{equation}
\begin{aligned}
	&U = \begin{bmatrix} 	
 	u(\vec{x_i}, t) \\ 
 	\vdots \\
	\end{bmatrix} _{\vec{x_i} \in X}  & \\
 	&\frac{\partial U}{\partial \tau} = \mathcal{O}_{BS}^\mathcal{T} U  &x_i \in X_{in}  \\	
 	&U = \phi(\vec{x_i}, \tau)  &x_i \in X_{far} \\
 	&U = 0 	&x_i \in X_{close} \\
 	\end{aligned} 	 
\end{equation}
Discretize the time $\vec{\tau} = \{\tau_m \}_{m = 1} ^ M$ with time step $ k_m = \tau_{m+1} - \tau_m$. The PDE is then expanded as:
\begin{gather}
U^{m+2} - \frac{4}{3}U^{m+1} + \frac{1}{3}U^{m} = \Delta \tau_m\mathcal{O}_{BS}^\mathcal{T} U^{m+2}  \qquad x_i \in X_{in} \\
\Leftrightarrow (I -\Delta \tau_m\mathcal{O}_{BS}^\mathcal{T} )U^{m+2} =\frac{4}{3}U^{m+1} - \frac{1}{3}U^{m} \qquad x_i \in X_{in} \\
\Leftrightarrow U^{m+2} = (I -\Delta \tau_m\mathcal{O}_{BS}^\mathcal{T})^{-1}(\frac{4}{3}U^{m+1} - \frac{1}{3}U^{m}) \qquad x_i \in X_{in} \\
\end{gather}
To eliminate the boundary conditions, the matrices are expanded to size $N \times N$. This new matrix, $C$, has identical rows to $(I -\Delta \tau_m\mathcal{O}_{BS}^\mathcal{T})$ for the indices corresponding with the interior points while being the identity matrix for the boundary rows. The boundary condition is then enforced by applying the respective condition to the boundary indices of the right-hand side. $U_{RHS} = (\frac{4}{3}U^{m+1} - \frac{1}{3}U^{m})$. This is done before solving the linear system. The final system is then given by.
\begin{equation}
\begin{gathered}
U^{m+2} = C^{-1} U_{RHS} \qquad \vec{x_i} \in X \\
\begin{cases}
U_{RHS} = \frac{4}{3}U^{m+1} - \frac{1}{3}U^{m}, & \vec{x_i}  \in X_{in}\\
U_{RHS} = 0, & \vec{x_i}  \in X_{close}\\
U_{RHS} = \phi(\vec{x_i},\tau_{m+1}), & \vec{x_i}  \in X_{far}
\end{cases}
\end{gathered}
\end{equation}

\section*{Algorithm}
\begin{algorithm}[H]
    \centering
    \caption{Pseudo code for solving BS}
    \begin{algorithmic}[1]
    		\State Scale ${X}_{eval}, K$ to and define problem on $[0, 1]^d]$
    		\State Generate center points in transformed coordinates
    		\State Obtain center points in standard system
    		\State Find points corresponding to close and far boundary. 
		\State Define $\mathbf{A}$ matrices, equation (\ref{ekv:A0}) - (\ref{ekv:A12}).     
        \State With Matrices $A$, build the rotated local matrices $B$. Equation \ref{ekv:BTransform}.
        \State Build the transform Black-Scholes operator $\mathcal{O}_{BS}$. Equation (\ref{ekv:OTransform}).
        \State Extended $\mathcal{O}_{BS}$ to form matrix $C$, eliminating BC. $U_{m+1} = U_{m} = u_0$
        \State Apply initial condition 
        \For {$\tau_m$; m++}
        \State $U_{temp} = C^{-1}U_{RHS}$
        \State $U_{RHS} = \frac{4}{3}*U_{m+1} - \frac{1}{3}*U_{m}$
        \State Apply boundary conditions: 
        \State $\qquad U_{RHS}(\vec{x}\in X_{far}, \tau_{m+1}) = \phi(\vec{x}, \tau_m)$
        \State $\qquad U_{RHS}(\vec{x}\in X_{close}, \tau_{m+1}) = 0$
        \State Move solution along:
        \State $\qquad U_{m} = U_{m+1}$
        \State $\qquad U_{m+1} = U_{m+2}$
        \State $\qquad U_{m+2} = U_{temp}$
        \EndFor
    \State Build evaluation matrix and evaluate solution in $X_{eval}$. Equation (\ref{ekv:EvalForm})
    \State Rescale problem
    \end{algorithmic}
        \label{alg:1}
\end{algorithm}


\section*{Generalization}

\subsection*{Repruducing Kernal}


\section{Resultat}

\begin{table}[H]
  \caption{Comparison between solutions generated with different maxOrder. $U_{d,maxOrder}$}
  \label{tab:rand_reverse}
  \centering
  \begin{tabular}{ccccccc}
     \toprule
	 $\mathbf{d}$ & $\mathbf{\Omega(X_{eval})}$ & $\mathbf{N_e}$ & $\mathbf{||U_{d,1} - U_{d,2}||_\infty}$&$\mathbf{||U_{d,2} - U_{d,3}||_\infty}$ & $\mathbf{||U_{d,1} - U_{d,3}||_\infty}$ & $\mathbf{||U_{d,1}||_\infty}$ \\
	 \midrule
     2 & $[ \frac{1}{3}K,\frac{5}{3}K ]^d $ & 121 & 0.0919 & N/A & N/A & 13.4848 \\
     3& $[ \frac{1}{3}K,\frac{5}{3}K ]^d $ &1331 & 0.0905 & 0.0727 & 0.1449 &  13.4103 \\
     4 & $[ \frac{1}{3}K,\frac{5}{3}K ]^d $ &625 & 0.0660 & 0.0622 & 0.1210 & 13.2346\\
     4 & $[ 0,2K ]^d $  &625 & 0.1906 & 0.1073 & 0.2581  &  20.1911\\
     5& $[ \frac{1}{3}K,\frac{5}{3}K ]^d $ &3125 & 0.0630 & 0.0532 & 0.1031&  13.0483\\
     6& $[ \frac{1}{3}K,\frac{5}{3}K ]^d $ &15625 & 0.0598 & TBD? & TBD?&  13.4752
\\

    \bottomrule
  \end{tabular}
\end{table}

\begin{equation}
N = \sum_{i=0}^{maxOrder} \begin{pmatrix} d \\ i \end{pmatrix} n^i
\end{equation}

\begin{table}[H]
  \caption{Run times for different dimensional problems and max order}
  \label{tab:runtime}
  \centering
  \begin{tabular}{cccccc}
     \toprule
	   &  &  &\multicolumn{3}{c}{\textbf{Run Time [s]}} \\
	 \cmidrule(lr){4-6}
	 \textbf{n} & $\mathbf{N_e}$ &\textbf{d} & $\mathrm{MaxOrder} = 1$ & $\mathrm{MaxOrder} = 2$ & $\mathrm{MaxOrder} = 3$ \\
	 \midrule
     10 & 1 &2 & 0.0197794 & 0.038682 & \textbf{-}\\
     10 & 1 &3 & 0.0251483 & 0.10886 & 2.6737 \\
     10 & 1 &4 & 0.0107194 & 0.54458 & 69.8605\\
     10 & 1 &5 & 0.0047781 & 2.2029 & 3579.4775\\  
     10 & 1 &6 & 0.0142638 & 6.3872 & -\\
     10 & 1 &7 & 0.0099809 & 16.9169 & -\\
     10 & 1 &8 & 0.0101846 & 40.2869 & -\\
     10 & 1 &9 & 0.0161873 & 86.2708 & -\\
     10 & 1 &10 & 0.0210623 & 175.614 & -\\
     10 & 1 &11 & 0.0249659 & 421.7462 & -\\
     10 & 1 &12 & 0.0293936 & 1491.1291 & -\\
     10 & 1 &13 & 0.0406836 & 3580.1538 & -\\
     10 & 1 &14 & 0.0507523 & - & -\\
     10 & 1 &15 & 0.0693409 & - & -\\
     10 & 1 &16 & 0.0961076 & - & -\\
     10 & 1 &17 & 0.0936543 & - & -\\
     10 & 1 &18 & 0.1303859 & - & -\\
     10 & 1 &19 & 0.2148575 & - & -\\
     10 & 1 &20 & 0.2233701 & - & -\\
     10 & 1 &21 & 0.2683903 & - & -\\
     10 & 1 &22 & 0.3102177 & - & -\\
     10 & 1 &23 & 0.2683903 & - & -\\
     10 & 1 &24 & 0.3102177 & - & -\\
     10 & 1 &25 & 0.3495040 & - & -\\
    \bottomrule
  \end{tabular}
\end{table}


\begin{figure}[H]
\centering
    \begin{tikzpicture}
    \begin{loglogaxis}[
        xmin=2,
        xmax=40,
        grid = major,
        xlabel=Dimensions (1),
        ylabel=Time (s),
        legend style={at={(1.2,1)},
        anchor=north,legend columns=1},
     ]
     %\addplot[
     %domain=1:20,
     %color=green,
     %]
     %{0.1*exp(x)}
     %;
     \addplot[
     color=red,
     mark=pentagon*,
     ] 
    table [x index = 0,y index= 1] {Data/time.txt}
    ;
     \addplot[
     color=black,
     mark=diamond*,
     ]
     table [x index= 4,y index= 5]{Data/time.txt}
     ;
     \addplot[
     color=blue,
     mark=*,
     ] 
    table [x index= 6 ,y index= 7] {Data/time.txt}
    ;
    \addplot[
    color = red,
    domain = 2:30,]
    {exp(-13.3667)*x^3.9899}
    coordinate [pos =0.5] (A)
   	coordinate [pos =0.95]  (B)
    ;
    \addplot[
    color = black,
    domain = 2:15,]
    {exp(-8.72)*x^(6.165)}
    coordinate [pos =0.85] (C)
   	coordinate [pos =0.95]  (D)
    ;
    \addplot[
    color = blue,
    domain = 2:7,]
    {exp(-14.5869)*x^(13.964)}
    coordinate [pos =0.8] (E)
   	coordinate [pos =0.95]  (F)
    ;
    \draw (A) -| (B) 
     node [pos=0.75,anchor=west] {3.99};
    \draw (C) -| (D) 
    node [pos=0.75,anchor=west] {6.16};
    \draw (E) -| (F) 
     node [pos=0.75,anchor=west] {13.96};
     
    \legend{MaxOrder = 1,MaxOrder = 2, MaxOrder = 3}
    \end{loglogaxis}
    \end{tikzpicture}
    \caption{Loglog time with fitted linjes }
    \label{fig:timeloglog}
\end{figure}

\begin{figure}
\centering
\begin{tikzpicture}
\begin{axis}[
xlabel=$S_1$,
ylabel=$S_2$,
]
\addplot[
color = red,
only marks,]
table [x index=0,y index=1,]{Data/XCentD2M1.txt};
\end{axis}
\end{tikzpicture}
\caption{Dim2 max1}
\end{figure}
\begin{figure}
\centering
\begin{tikzpicture}
\begin{axis}[
xlabel=$S_1$,
ylabel=$S_2$,
]
\addplot[
color = red,
only marks,]
table [x index=0,y index=1,]{Data/XCentD2M2.txt};
\end{axis}
\end{tikzpicture}
\caption{Dim2 max1}
\end{figure}
\begin{figure}
\centering
\begin{tikzpicture}
\begin{axis}[
xlabel=$S_1$,
ylabel=$S_2$,
zlabel=$S_3$,
]
\addplot3[
color = red,
only marks,]
table [x index=0,y index=1,z index = 2]{Data/XCentD3M1.txt};
\end{axis}
\end{tikzpicture}
\caption{Dim2 max1}
\end{figure}
\begin{figure}
\centering
\begin{tikzpicture}
\begin{axis}[
xlabel=$S_1$,
ylabel=$S_2$,
zlabel=$S_3$,
]
\addplot3[
color = red,
only marks,]
table [x index=0,y index=1,z index = 2]{Data/XCentD3M2.txt};
\end{axis}
\end{tikzpicture}
\caption{Dim3 max2}
\end{figure}
\begin{figure}
\centering
\begin{tikzpicture}
\begin{axis}[
xlabel=$S_1$,
ylabel=$S_2$,
zlabel=$S_3$,
]
\addplot3[
color = red,
only marks,]
table [x index=0,y index=1,z index = 2]{Data/XCentD3M3.txt};
\end{axis}
\end{tikzpicture}
\caption{Dim3 max3}
\end{figure}

\begin{figure}[H]
\centering
    \begin{tikzpicture}
    \begin{loglogaxis}[
        xmin=1,
        xmax=45,
        grid = major,
        xlabel=Dimensions (1),
        ylabel=Total Centerponints (s),
        legend style={at={(1.2,1)},
        anchor=north,legend columns=1},
     ]
     \addplot[
     color=red,
     mark=pentagon*,
     ] 
    table [x index = 0,y index= 1] {Data/NIncrese.txt}
    ;
     \addplot[
     color=black,
     mark=diamond*,
     ]
     table [x index= 0,y index= 2]{Data/NIncrese.txt}
     ;
     \addplot[
     color=blue,
     mark=*,
     ] 
    table [x index= 3 ,y index= 4] {Data/NIncrese.txt}
    ;
    \addplot[
    color = black,
    domain = 1:35,]
    {exp(3.4655)*x^(2.1421)}
    coordinate [pos =0.8] (C)
   	coordinate [pos =0.95]  (D)
    ;
    \addplot[
    color = red,
    domain = 1:35,]
    {exp(2.3490)*x^(0.9856)}
    coordinate [pos =0.8] (A)
   	coordinate [pos =0.95]  (B)
    ;
    \addplot[
    color = blue,
    domain = 1:35,]
    {exp(3.7718)*x^(3.4152)}
    coordinate [pos =0.8] (E)
   	coordinate [pos =0.95]  (F)
    ;
    \draw (A) -| (B) 
     node [pos=0.75,anchor=west] {0.9856};
    \draw (C) -| (D) 
    node [pos=0.75,anchor=west] {2.1421};
    \draw (E) -| (F) 
     node [pos=0.75,anchor=west] {3.4152};
    \legend{MaxOrder = 1,MaxOrder = 2, MaxOrder = 3}
    \end{loglogaxis}
    \end{tikzpicture}
    \caption{Total number of center points (N) for n = 10 }
    \label{fig:NIncrese}
\end{figure}
\begin{figure}
\centering
\begin{tikzpicture}
\begin{axis}[colorbar,
        colormap/jet,
        colorbar style={
        ytick={0,3,6,9,12,15},}]
\addplot3
[surf,shader=faceted]
table [x index = 0, y index = 1, z index = 2]{Data/SolutionD2M2.txt};
\end{axis}
\end{tikzpicture}
\caption{Computed solution around strike. Two dimensions and Max order 2}
\end{figure}


\begin{figure}
\centering
\begin{tikzpicture}
\begin{axis}[colorbar,
        colormap/jet,
        colorbar style={
        ytick=data,}]
\addplot3
[surf,shader=faceted]
table [x index = 0, y index = 1, z index = 2]{Data/ErrorD2M2.txt};
\end{axis}
\end{tikzpicture}
\caption{Error around strike from solution compared with established solver. Two dimensions and Max order 2}
\end{figure}

\begin{figure}
\centering
\begin{tikzpicture}
\begin{axis}[colorbar,
        colormap/jet,
        	title=\textbf{Solution at Evalutaion Points, Center Points 			visiualized in red},
        view={-45}{45},
        xlabel=$s_1$,
        ylabel=$s_2$,
        colorbar style={
        ytick=data,}]
\addplot3
[surf,shader=faceted]
table [x index = 0, y index = 1, z index = 2]{Data/D2M1NoRot.txt}
;

\addplot3[
mark=none,
color=red,
line width=3pt,
] 
coordinates {(5,16,10) (35,16,10)}
;
\addplot3[
mark=none,
color=red,
line width=3pt,
] 
coordinates {(16,5,10) (16,35,10)};
\end{axis}
\end{tikzpicture}
\caption{Solution around strike from solution. No rotation D2M1 T = 1}
\end{figure}


\begin{figure}
\centering
\begin{tikzpicture}
\begin{axis}[colorbar,
        colormap/jet,
        colorbar style={
        ytick=data,}]
\addplot3
[surf,shader=faceted]
table [x index = 0, y index = 1, z index = 2]{Data/ErrorNoRot.txt}
;

\addplot3[
mark=none,
color=red,
line width=3pt,
] 
coordinates {(5,20,5) (30,20,5)}
;
\addplot3[
mark=none,
color=red,
line width=3pt,
] 
coordinates {(20,5,5) (20,30,5)};
\end{axis}
\end{tikzpicture}
\caption{Error around strike from solution. No rotation D2M1 T = 1}
\end{figure}

\begin{figure}
\centering
\begin{tikzpicture}
\begin{axis}[colorbar,
        colormap/jet,
        colorbar style={
        ytick=data,}]
\addplot3
[surf,shader=faceted]
table [x index = 0, y index = 1, z index = 2]{Data/SolutionRot.txt}
;

\addplot3[
mark=none,
color=red,
line width=3pt,
] 
coordinates {(5,35,15) (35,5,15)}
;
\addplot3[
mark=none,
color=red,
line width=3pt,
] 
coordinates {(7,7,15) (34,34,15)};
\end{axis}
\end{tikzpicture}
\caption{Solution around strike from solution. Rotation D2M1 T = 1}
\end{figure}

\begin{figure}
\centering
\begin{tikzpicture}
\begin{axis}[colorbar,
        colormap/jet,
        colorbar style={
        ytick=data,}]
\addplot3
[surf,shader=faceted]
table [x index = 0, y index = 1, z index = 2]{Data/ErrorRot.txt}
;

\addplot3[
mark=none,
color=red,
line width=3pt,
] 
coordinates {(5,35,0.5) (35,5,0.5)}
;
\addplot3[
mark=none,
color=red,
line width=3pt,
] 
coordinates {(7,7,0.5) (34,34,0.5)};
\end{axis}
\end{tikzpicture}
\caption{Solution around strike from solution. Rotation D2M1 T = 1}
\end{figure}

\begin{figure}[H]
\centering
	\begin{tikzpicture}
    \begin{loglogaxis}[
        xmin=5,
        xmax=100,
        grid=major,
        xlabel= $n$-Points in each dimention,
        ylabel=$||\mathrm{error}||_\infty$- Absolute inf-norm error ,
        legend style={at={(1,1)},
        anchor=north east,legend columns=1},
     ]
     \addplot[
     color=blue,
     mark=*,
     ] 
    table [x index= 0 ,y index= 1] {Data/Space_converge.txt}
    ;
    \legend{Datapoints}
    \end{loglogaxis}
    \end{tikzpicture}
\end{figure}

\begin{figure}[H]
\centering
	\begin{tikzpicture}
    \begin{loglogaxis}[
        xmin=2,
        title=\textbf{Time Convergence in loglog graph},
        xmax=40,
        grid=major,
        xlabel= $M$- number of timesteps,
        ylabel=$||\mathrm{error}||_\infty$- Absolute inf-norm error ,
        legend style={at={(1,1)},
        anchor=north east,legend columns=1},
     ]
     \addplot[
     color=red,
     mark=*,
     ] 
    table [x index= 0 ,y index= 1] {Data/conv_time.txt}
    ;
    \legend{Datapoints}
    \end{loglogaxis}
    \end{tikzpicture}
\end{figure}

\begin{gather*}
K = 20, \; T = 0.5, \; r = 0.02, \\
\\
C = \begin{pmatrix}
\sigma_1^2 & \rho_{12}\sigma_1 \sigma_2 \\
\rho_{12}\sigma_2 \sigma_1 & \sigma_2^2
\end{pmatrix}
= 
\begin{pmatrix}
0.0225 & 0.0150 \\
0.0150 & 0.0400 
\end{pmatrix}
\\ 
\end{gather*}

\begin{gather*}
n = 40, \; M = 15, \; \epsilon = 50 \\
 \mathrm{anchor} = 21 
\end{gather*}


\begin{figure}[H]
\centering
\begin{tikzpicture}
\begin{axis}[
xlabel=$S_1$,
ylabel=$S_2$,
]
\addplot[
color = red,
only marks,]
table [x index=2,y index=3,]{Data/XochXT.txt};
\addplot[
color = blue,
only marks,]
table [x index=6,y index=7,]{Data/XochXT.txt};
\end{axis}
\end{tikzpicture}
\end{figure}

\begin{figure}
\centering
\begin{tikzpicture}
\begin{axis}[colorbar,
        colormap/jet,
        colorbar style={
        ytick=data,}]
\addplot3
[surf,shader=faceted]
table [x index = 0, y index = 1, z index = 2]{Data/FinalSolution.txt}
;

\addplot3[
mark=none,
color=red,
line width=3pt,
] 
coordinates {(5,35,14) (35,5,14)}
;
\addplot3[
mark=none,
color=red,
line width=3pt,
] 
coordinates {(7,7,14) (34,34,14)};
\end{axis}
\end{tikzpicture}
\caption{Solution around strike from solution. Rotation D2M1 T = 1}
\end{figure}


\begin{figure}
\centering
\begin{tikzpicture}
\begin{axis}[colorbar,
        colormap/jet,
        view={-45}{45},
        title=\textbf{Error for problem: $d=2$, $m=1$},
        xlabel=$s_1$,
        ylabel=$s_2$,
        colorbar style={
        ytick=data,}]
\addplot3
[surf,shader=faceted]
table [x index = 0, y index = 1, z index = 2]{Data/FinalFiner.txt}
;
\end{axis}
\end{tikzpicture}
\caption{Error final}
\end{figure}

\begin{figure}
\centering
\begin{tikzpicture}
\begin{axis}[colorbar,
        colormap/jet,
        	title=\textbf{True Solution},
        view={-45}{45},
        xlabel=$s_1$,
        ylabel=$s_2$,
        colorbar style={
        ytick=data,}]
\addplot3
[surf,shader=faceted]
table [x index = 0, y index = 1, z index = 2]{Data/True.txt}
;
\end{axis}
\end{tikzpicture}
\caption{True solution. For comparing with unrotated}
\end{figure}

\end{document}
